%%%Tema para beamer "IM-UNAM", versión 2.0
%%%Desarrollado por Ernesto Vázquez (shino@ciencias.unam.mx)
%%%IM-UNAM. Agosto 2021.
%%%Basado en: 
%%Tema para beamer "Imunam", versión 1.0
%%Desarrollado por Geri Morales (gerino.morales@im.unam.mx)
%%Imate Cuernavaca, UNAM. Agosto 2014.NAM. Agosto 2014.
\documentclass{beamer}
\usepackage[T1]{fontenc}
\usepackage{kpfonts}
%\usepackage{iwona}

\usepackage[utf8]{inputenc}
\usepackage[spanish,mexico]{babel}

%Para hacer bloques de diferentes colores
\newenvironment{variableblock}[3]{
\setbeamercolor{block body}{#3}
\setbeamercolor{block title}{#2}
\begin{block}{#1}}{\end{block}}

\newenvironment{obs}{\begin{variableblock}{Observación}{bg=red,fg=white}{fg=black}}{\end{variableblock}}

\newenvironment{ej}{\begin{variableblock}{Ejemplo}{bg=gray,fg=white}{fg=black}}{\end{variableblock}}

\newenvironment{df}{\begin{variableblock}{Definición}{bg=orounam,fg=white}{fg=black}}{\end{variableblock}}

\newenvironment{cor}{\begin{variableblock}{Corolario}{bg=purple,fg=white}{fg=black}}{\end{variableblock}}

\newenvironment{prop}{\begin{variableblock}{Proposición}{bg=azulunam,fg=white}{fg=black}}{\end{variableblock}}

\newenvironment{lema}{\begin{variableblock}{Lema}{bg=purple,fg=white}{fg=black}}{\end{variableblock}}

\newenvironment{teo}{\begin{variableblock}{Teorema}{bg=azulunam,fg=white}{fg=black}}{\end{variableblock}}


\usetheme{Imunam} 

\title{Título de la presentación}
\author{Nombre Apellido \\ %Nombre del autor
\texttt{fulano@im.unam.mx}}
	      
\date{\today}

%\institute{Instituto de Matemáticas}
%Dado que el texto "Intituto de matemáticas" aparece en el logo, parece redundante incluirlo además con éste comando.

\begin{document}

\begin{frame}
\titlepage %Necesario para generar la portada
\end{frame}

%La siguiente diapositiva es opcional, si se quiere la tabla de contenidos (Se sebe compilar dos veces el documento para que funcione)
\begin{frame}
\tableofcontents %Imprime la tabla de contenido
\end{frame}

\section{Introducción} %Título de la sección
\begin{frame}
\frametitle{Título de la diapositiva}
\framesubtitle{Subtítulo} %Subtítulo de la diapositiva
Algo de texto en la parte superior
$$ \int_{0}^{\infty} \frac{5x^2}{\sqrt{a+b}}\, dx$$

\begin{itemize}
\item[\checkmark] Un elemento en la lista
\item Segundo elemento
\item Otro elemento
\item Y finalmente, otro más
\end{itemize}
\end{frame}

\begin{frame}{Otra diapositiva}
Texto adicional. No tiene un propósito en particular más que ocupar algo de espacio.
Texto adicional. No tiene un propósito en particular más que ocupar algo de espacio.
Texto adicional. No tiene un propósito en particular más que ocupar algo de espacio.
Texto adicional. No tiene un propósito en particular más que ocupar algo de espacio.

$\sum_0^{\infty} a_i$

Texto adicional. No tiene un propósito en particular más que ocupar algo de espacio.
\end{frame}

\section{Segunda sección} %%Otra sección
\begin{frame}
\frametitle{Un teorema}
\begin{teo}
Los números primos son infinitos
\end{teo}

\begin{obs}
Mira la observación
\end{obs}

Y texto

\begin{ej}
Un ejemplo de ejemplo.
\end{ej}
Y algo de texto
\end{frame}


\begin{frame}{Otros ejemplos}
Más texto.

\begin{df}
Definición de algo.
\end{df}

\begin{cor}
Consecuencia de un resultado.
\end{cor}

\begin{prop}
Propuesta de proposición.
\end{prop}
\end{frame}

\begin{frame}{Últimos ejemplos}

\begin{lema}
Un resultado importante.
\end{lema}

y el último resultado:

\begin{teo}
Teorema principal
\end{teo}

\end{frame}

\end{document}